\documentclass[a4paper,11pt]{article}
\usepackage[utf8]{inputenc}
\usepackage[T1]{fontenc}
\usepackage{lmodern}

\usepackage{microtype}
\usepackage[english]{babel}
\usepackage[top=3cm, bottom=3.5cm, left=3cm, right=3cm]{geometry}
\usepackage[citecolor=black]{hyperref}
%\usepackage[normalem]{ulem}
\usepackage{hyphenat}
\usepackage{parskip}
\usepackage{amsmath}
\usepackage{amssymb}
\usepackage{mathtools}

\hypersetup{urlcolor=blue, colorlinks=true, hypertexnames=true}
\let\origUrlBreaks\UrlBreaks
\renewcommand*{\UrlBreaks}
{\origUrlBreaks\do\a\do\b\do\c\do\d\do\e\do\f\do\g\do\h\do\i\do\j\do\k\do\l\do\m\do\n\do\o\do\p\do\q\do\r\do\s\do\t\do\u\do\v\do\w\do\x\do\y\do\z\do\A\do\B\do\C\do\D\do\E\do\F\do\G\do\H\do\I\do\J\do\K\do\L\do\M\do\N\do\O\do\P\do\Q\do\R\do\S\do\T\do\U\do\V\do\W\do\X\do\Y\do\Z}

\newcommand*\N{\mathbb{N}}
\newcommand*\Z{\mathbb{Z}}
\newcommand{\Q}{{\mathbb Q}}
\newcommand{\R}{{\mathbb R}}
\newcommand*\ptcut{\textsf{PtCut}}
\newcommand*\eps{\ensuremath{\varepsilon}}



\title{This is far from a documentation for \ptcut}

\author{Christoph L\"uders}


\begin{document}

\maketitle

\abstract{\ptcut\ is a Python program to calculate tropical prevarieties and tropical equibrilations.
This is its documentation.  Well, it's a sad excuse for one.
}




\section{Command line options}

\begin{itemize}
\item \texttt{-e}$X,\ldots$, \texttt{-{}-ep}~$X,\ldots$\par
Specify the reciprocal of the \eps\ parameter.  Default is \texttt{-e5}, i.e., $\eps = 1/5$.
You can specify integers or fixed point float values (like \texttt{-e1.1}).
The value \texttt{-{}-ep~p}\emph{X} signifies the smallest prime number larger that $10^X$,
i.e., \texttt{p1} is 11, \texttt{p2} is 101, \texttt{p3} is 1009, etc.
The value \texttt{-{}-ep~c}\emph{X} signifies the smallest prime number larger that $10^X$ plus 1,
a composite number.
The value \texttt{-{}-ep~e}\emph{X} signifies $10^X$.

You can specify multiple values for $\eps$, separated by commas (but no spaces).
Calculations will then be performed for each values successively.

\emph{Note: all those values specify the reciprocal of $\eps$!}

\emph{Note: there is only one `\texttt{-}' and no space for \texttt{-e}!}

\emph{Note: there must be no spaces between different values for $\eps$!}

\item \texttt{-{}-seed~$X$}\par
Specify the random seed that is e.g.\ used for \texttt{-{}-filter} and \texttt{-{}-shuffle}.
If not specified, a seed by the operating system is used.
Specifying the same seed will lead to the same random choices.
The seed used is printed prior to calculation.

\item \texttt{-{}-st}\par
Switch on timing of intersections and inclusion tests.  Can cost some 20\% run-time.

\item \texttt{-s}\par
Switch on extensive logging of intersections and inclusion tests.
This can slow calculation down by a large factor.

\item \texttt{-p}\par
Print the solution in \texttt{.lp} format to the screen.

\item \texttt{-{}-stl}\par
Print the solution in \texttt{.lp} format to the log file.

\item \texttt{-{}-no}~$X,Y,\ldots$\par
Remove model(s) \emph{X}, \emph{Y}, etc.\ from list of models to calculate.
Enables you to use \texttt{-{}-all -{}-no BIOMD0000000026} to calculate all models,
but Biomodel~28.

\emph{Note: there must be no spaces between model names!}

\item \texttt{-{}-sumup}, \texttt{-{}-sum}\par
When handling equations, substituting the parameters will re-evaluate the formulas
and collect terms with matching variables.

Example: assume $k_1=2$, $k_2=3$.
Then $k_1 x_1 + k_2 x_1$ becomes
$2 x_1 + 3 x_1$ (intermediate) and $5 x_1$ in the end.  Even more dramatic, $k_1 x_1 - k_2 x_1$
will collapse to $0$ if $k_1 = k_2$.

Default is to switch off the \texttt{sumup} feature, thus creating multiple points after
tropicalization.
\texttt{-{}-sumup} requires \texttt{-{}-keep-coeff}.

\item \texttt{-{}-keep-coeff}, \texttt{-{}-keep-coeffs}\par
When handling the equations, literal parameters will be not be ignored
(parameters listed in \texttt{Params.txt} are never ignored).

Example: in the term $2 k_1 x_1$, the ``2'' will not be ignored (the value of $k_1$
is never ignored).

Default is to ignore literal parameters.

\item \texttt{-{}-merge-param}, \texttt{-{}-merge-params}\par
When handling the equations, just substitute the parameters and then do the logarithm and rounding.

By default, the logarithm and rounding are calculated per parameter separately,
thus rounding errors might add up.

\item \texttt{-C}\par
Switch into ``Chris mode'', that is a shorthand for
\texttt{-{}-sumup}, \texttt{-{}-keep-coeff}, \texttt{-{}-merge-params}.

\item \texttt{-{}-bb}, \texttt{-{}-bbox}\par
Switch on bounding boxes: in common planes checking, \ptcut\ will calculate
bounding boxes for all polyhedra. These individual bounding boxes are then
joined to form the bounding box of their bag. The bag bounding boxes of all bags
are then intersected.  This total bounding box is the space in which all solutions
must lie and thus, all polyhedra are intersected with it.

This can dramatically reduce the number of possible combinations.  However,
the calculation needs V-representation and it takes time to compute that.
This means that it's not always efficient to calculate bounding boxes.

\item \texttt{-{}-filter~$X$}\par
After each iteration, limit the number of temporary polyhedra to $X$.
If there are more than $X$ polyhedra, a random selection of them is discarded
(actually, they are not calculated in the first place).  Obviously, this
saves a lot of time, but computes only a subset of the solution.

\item \texttt{-{}-remove~$X, \ldots$}\par
Remove named polyhedra from bags prior to calculation. The respective names
have the form $N.M$, where $N$ is the number of the bag (starting at $0$)
and $M$ is the number of the polyhedron in the bag (starting at $0$).
Those are the same numbers that are listed when common planes
calculation finds superfluous polyhedra and lists them as \texttt{Removed:}
(see log file or use \texttt{-v} option to see this output on the screen).

If only polyhedra are removed that were listed by common planes calculation,
this will \emph{not} change the solution. So, one can run \ptcut\
multiple times, with and without certain parameters (like \texttt{-{}-box},
\texttt{-{}-nc}, \texttt{-{}-one}, \texttt{-{}-common2}, different \texttt{-a}
or \texttt{-l} values) and collect superfluous polyhedra and use them
with \texttt{-{}-remove} to reduce the number of combinations.

\emph{Note: there must be no spaces between polyhedra names!}

\item \texttt{-{}-nc}\par
Switch off common places calculation.  Sometimes common planes
calculation causes polyhedra to be too complex and calculations can get
extremely slow.

\item \texttt{-{}-one}\par
Only bags with one polyhedron are intersected into all other bags
while calculating common planes.

\item \texttt{-{}-common2}\par
Only bags with one polyhedron and common hyperplanes (but no common half-spaces)
are intersected into all other bags
while calculating common planes.

\item \texttt{-{}-shuffle}\par
Randomly shuffle the order of bags at the beginning.
If a random order of intersections is desired, you must add \texttt{-l0}
option as well to switch off \emph{likeness} calculations when selecting
the next bag for intersection.

\item \texttt{-{}-runs~$X$}\par
Perform $X$ runs of the same model(s).
This makes sense when used with \texttt{-{}-shuffle} and \texttt{-{}-maxruntime},
since each time a different
order will be used.

\item \texttt{-{}-maxruntime~$X$}\par
Stop computation after $X$ seconds. Useful with \texttt{-{}-shuffle} and \texttt{-{}-runs}.

\item \texttt{-{}-all}\par
\texttt{-{}-simple}\par
\texttt{-{}-easy}\par
\texttt{-{}-fast}\par
\texttt{-{}-slow}\par
\texttt{-{}-hard}\par
Select a list of models to calculate. There are certain relations:
\begin{align*}
\textsf{simple} &= \textsf{fast} + \textsf{slow}  \\
\textsf{all} &= \textsf{simple} + \textsf{hard}  \\
\textsf{all} &\supset \textsf{simple} \supset \textsf{easy} \supset \textsf{fast}
\end{align*}

\item \texttt{-{}-nonewton}\par
Build no Newton polytope to compute and filter out interior points.
Instead, just use all points, since superfluous points will be removed anyway in the process.

\item \texttt{-{}-cc}, \texttt{-{}-concomp}\par
Calculate the number of connected components of the polyhedra in the prevariety.
See \texttt{-{}-contype~$X$} for what exactly ``connected'' means.

\item \texttt{-{}-contype~$X$}\par
When are two polyhedra considered ``connected''? Several types are supported:

\begin{itemize}
\item Type 0: The intersection of both polyhedra is non-empty.
\item Type 1: The intersection of both polyhedra is non-empty and its dimension is lower than both polyhedra, or a point.
\item Type 2: Both polyhedra have the same dimension and their intersection is non-empty.
\end{itemize}

\item \texttt{-{}-multiple-lpfiles}\par
Instead of saving the solution into one \texttt{.lp} file, separated by lines starting with 40 backslashes
(multi-lp files), save each polyhedron in its own \texttt{.lp} file.

\end{itemize}

%\clearpage
%\bibliographystyle{myalpha}
%\bibliography{../../chris-all}

\end{document}
